\documentclass[a4,12pt]{article}
\usepackage[utf8]{inputenc}
\usepackage[T2A]{fontenc}
\usepackage[english, russian]{babel}


\usepackage{csquotes}
\usepackage{epigraph}
\usepackage{tcolorbox}
% Определяем окружение для Формата ввода
\newtcolorbox{inputformat}[1][]{colback=blue!5!white, colframe=blue!65!black,
    fonttitle=\bfseries, title=Формат ввода, #1}

% Определяем окружение для Формата вывода
\newtcolorbox{outputformat}[1][]{colback=green!5!white, colframe=green!65!black,
    fonttitle=\bfseries, title=Формат вывода, #1}

% Определяем окружение для комментария
\newtcolorbox{exercisecomment}[1][]{colback=yellow!5!white, colframe=yellow!80!black, fonttitle=\bfseries, title=Комментарий, #1}

% Определяем окружение для замечания
\newtcolorbox{exercisenote}[1][]{colback=red!5!white, colframe=red!80!black, fonttitle=\bfseries, title=Замечание, #1}


\usepackage{geometry}
%\geometry{papersize={20.3 cm,25.4 cm}}
\geometry{left=2cm}
\geometry{right=2cm}
\geometry{top=2cm}
\geometry{bottom=2cm}


\usepackage{amsmath, amsfonts, amsthm, amssymb, amsopn, amscd}
\usepackage{enumerate}
\usepackage{enumitem}
\usepackage[mathscr]{eucal}

\usepackage{hyperref}
\hypersetup{unicode=true,final=true,colorlinks=true}

\theoremstyle{remark}
%\newtheorem{exercises}{Упражнения}
\newtheorem{exercise}{\textbf{Упражнение}}[section]
\renewcommand{\theexercise}{\textbf{\arabic{exercise}}}
%\renewcommand{\theexercise}{\textbf{\#\arabic{exercise}}}


\title{Листок 07. Функции в Python}

\author{А.Н. Лата}

\begin{document} 

%\maketitle

\section*{\centering Листок 07. Функции в Python}

\begin{exercisenote}[title=Замечания]
\begin{itemize}
    \item для решение приведенных ниже упражнений требуется создавать (определять) пользовательские функции.
    \item при решении упражнений полученные результаты выведите на экран с помощью функции {\color{blue}{print()}}.
    \item позаботесь о том, чтобы выводимый на экран результат был снабжен информацией о нем, там где это необходимо.
    \item не забывайте писать комментарии к вашему коду
\end{itemize}
\end{exercisenote}

\textbf{Упражнения}

\textbf{Функции с позиционными аргументами}

\begin{enumerate}
    \item Напишите функцию \textbf{sum\_two\_numbers}, принимающую 2 аргумента – числа a и b – и возвращающую 1 значение: их сумму.
    \item Напишите функцию \textbf{is\_even}, принимающую 1 аргумент – число n – и возвращающую 1 значение: \textbf{True}, если число чётное, и \textbf{False} в противном случае.
    \item Напишите функцию \textbf{max\_of\_three}, принимающую 3 аргумента – числа \textbf{a}, \textbf{b} и \textbf{c} – и возвращающую 1 значение: наибольшее из них. \textbf{Не используйте встроенную функцию max()}.
    \item Напишите функцию \textbf{is\_vowel}, принимающую 1 аргумент – символ \textbf{ch} – и возвращающую 1 значение: \textbf{True}, если он является гласной, и \textbf{False} в противном случае.
    \item Напишите функцию \textbf{count\_vowels}, принимающую 1 аргумент – строку \textbf{s} – и возвращающую 1 значение: количество гласных в этой строке.
    \item Напишите функцию \textbf{factorial}, принимающую 1 аргумент – число \textbf{n} – и возвращающую 1 значение: его факториал. \textbf{Решите задачу двумя способами: рекурсивно и итеративно.}
    \item Напишите функцию \textbf{reverse\_string}, принимающую 1 аргумент – строку \textbf{s} – и возвращающую 1 значение: её в обратном порядке.
    \item Напишите функцию \textbf{is\_palindrome}, принимающую 1 аргумент – строку \textbf{s} – и возвращающую 1 значение: \textbf{True}, если строка является палиндромом, и \textbf{False} в противном случае.
    \item Напишите функцию \textbf{is\_palindrome\_ignore\_spaces}, принимающую 1 аргумент – строку \textbf{s} – и возвращающую 1 значение: \textbf{True}, если строка является палиндромом, игнорируя регистр и пробелы, и \textbf{False} в противном случае.
    \item Напишите функцию \textbf{remove\_duplicates}, принимающую 1 аргумент – список \textbf{lst} – и возвращающую 1 значение: новый список без повторяющихся элементов.
    \item Напишите функцию \textbf{fibonacci}, принимающую 1 аргумент – число \textbf{n} – и возвращающую 1 значение: $n$-ое число Фибоначчи. \textbf{Решите задачу двумя способами: рекурсивно и итеративно.}
    \item Напишите функцию \textbf{sum\_list}, принимающую 1 аргумент – список чисел \textbf{lst} – и возвращающую 1 значение: их сумму. \textbf{При решении не используйте встроенные функции.}
    \item Напишите функцию \textbf{find\_min}, принимающую 1 аргумент – список чисел \textbf{lst} – и возвращающую 1 значение: минимальное значение в списке. \textbf{При решении не используйте встроенные функции.}
    \item Напишите функцию \textbf{find\_max}, принимающую 1 аргумент – список чисел \textbf{lst} – и возвращающую 1 значение: максимальное значение в списке. \textbf{При решении не используйте встроенные функции.}
    \item Напишите функцию \textbf{is\_prime}, принимающую 1 аргумент – число n – и возвращающую 1 значение: \textbf{True}, если число простое, и \textbf{False} в противном случае.
    \item Напишите функцию \textbf{sort\_list}, принимающую 1 аргумент – список чисел lst – и возвращающую 1 значение: список, отсортированный по возрастанию. \textbf{При решении не используйте встроенные функции.}
    \item Напишите функцию \textbf{average}, принимающую 1 аргумент – список чисел \textbf{lst} – и возвращающую 1 значение: их среднее значение.
    \item Напишите функцию \textbf{merge\_lists}, принимающую 2 аргумента – списки \textbf{lst1} и \textbf{lst2} – и возвращающую 1 значение: объединённый список.
    \item Напишите функцию \textbf{capitalize\_words}, принимающую 1 аргумент – строку \textbf{s} – и возвращающую 1 значение: строку, где каждое слово начинается с заглавной буквы.
    \item Напишите функцию \textbf{count\_occurrences}, принимающую 2 аргумента – список \textbf{lst} и элемент \textbf{elem} – и возвращающую 1 значение: количество вхождений элемента \textbf{elem} в список \textbf{lst}. \textbf{При решении не используйте встроенные функции и методы.}
    \item Напишите функцию \textbf{remove\_vowels}, принимающую 1 аргумент – строку \textbf{s} – и возвращающую 1 значение: строку без гласных букв.
    \item Напишите функцию \textbf{power}, принимающую 2 аргумента – число \textbf{base} и степень \textbf{exp} – и возвращающую 1 значение: результат возведения числа \textbf{base} в степень \textbf{exp}. \textbf{При решении не используйте встроенные функции.}
    \item Напишите функцию \textbf{swap\_first\_last}, принимающую 1 аргумент – список \textbf{lst} – и возвращающую 1 значение: список с первым и последним элементами, поменянными местами.
    \item Напишите функцию \textbf{count\_characters}, принимающую 1 аргумент – строку \textbf{s} – и возвращающую 1 значение: количество символов в строке.
    \item Напишите функцию \textbf{to\_uppercase}, принимающую 1 аргумент – строку \textbf{s} – и возвращающую 1 значение: строку в верхнем регистре.
    \item Напишите функцию \textbf{to\_lowercase}, принимающую 1 аргумент – строку \textbf{s} – и возвращающую 1 значение: строку в нижнем регистре.
    \item Напишите функцию \textbf{find\_second\_largest}, принимающую 1 аргумент – список чисел \textbf{lst} – и возвращающую 1 значение: второе по величине число.
    \item Напишите функцию \textbf{sum\_of\_squares}, принимающую 1 аргумент – список чисел \textbf{lst} – и возвращающую 1 значение: сумму их квадратов.
    \item Напишите функцию \textbf{find\_longest\_word}, принимающую 1 аргумент – строку \textbf{s} – и возвращающую 1 значение: самое длинное слово в строке.
    \item Напишите функцию \textbf{count\_words}, принимающую 1 аргумент – строку \textbf{s} – и возвращающую 1 значение: количество слов в строке.
    \item Напишите функцию \textbf{remove\_negatives}, принимающую 1 аргумент – список \textbf{lst} – и возвращающую 1 значение: новый список без отрицательных чисел.
    \item Напишите функцию \textbf{multiply\_list}, принимающую 1 аргумент – список чисел \textbf{lst} – и возвращающую 1 значение: произведение всех элементов списка.
    \item Напишите функцию \textbf{is\_anagram}, принимающую 2 аргумента – строки \textbf{s1} и \textbf{s2} – и возвращающую 1 значение: \textbf{True}, если одна строка является анаграммой другой, и \textbf{False} в противном случае.
    \item Напишите функцию \textbf{unique\_elements}, принимающую 1 аргумент – список \textbf{lst} – и возвращающую 1 значение: новый список только с уникальными элементами.
    \item Напишите функцию \textbf{is\_sublist}, принимающую 2 аргумента – списки \textbf{lst1} и \textbf{lst2} – и возвращающую 1 значение: \textbf{True}, если \textbf{lst1} является подсписком \textbf{lst2}, и \textbf{False} в противном случае.
    \item Напишите функцию \textbf{find\_median}, принимающую 1 аргумент – список чисел \textbf{lst} – и возвращающую 1 значение: медиану списка.
    \item Напишите функцию \textbf{remove\_evens}, принимающую 1 аргумент – список \textbf{lst} – и возвращающую 1 значение: новый список без чётных чисел.
    \item Напишите функцию \textbf{largest\_negative}, принимающую 1 аргумент – список \textbf{lst} – и возвращающую 1 значение: наибольшее отрицательное число в списке.
    \item Напишите функцию \textbf{count\_multiples\_of\_three}, принимающую 1 аргумент – список \textbf{lst} – и возвращающую 1 значение: количество чисел, кратных трём, в списке.
    \item Напишите функцию \textbf{join\_with\_comma}, принимающую 1 аргумент – список строк \textbf{lst} – и возвращающую 1 значение: строку с элементами, разделёнными запятой.
    \item Напишите функцию \textbf{power\_list}, принимающую 2 аргумента – список чисел \textbf{lst} и степень \textbf{exp} – и возвращающую 1 значение: новый список с элементами, возведёнными в степень \textbf{exp}.
    \item Напишите функцию \textbf{abs\_negatives}, принимающую 1 аргумент – список \textbf{lst} – и возвращающую 1 значение: список с абсолютными значениями отрицательных чисел.
    \item Напишите функцию \textbf{fahrenheit\_to\_celsius}, принимающую 1 аргумент – температуру \textbf{f} в градусах Фаренгейта – и возвращающую 1 значение: температуру в градусах Цельсия.
    \item Напишите функцию \textbf{is\_alpha}, принимающую 1 аргумент – строку \textbf{s} – и возвращающую 1 значение: \textbf{True}, если строка содержит только буквы, и \textbf{False} в противном случае.
    \item Напишите функцию \textbf{remove\_spaces}, принимающую 1 аргумент – строку \textbf{s} – и возвращающую 1 значение: строку без пробелов.
    \item Напишите функцию \textbf{fibonacci\_sequence}, принимающую 1 аргумент – число \textbf{n} – и возвращающую 1 значение: список из первых n чисел Фибоначчи.
    \item Напишите функцию \textbf{all\_even}, принимающую 1 аргумент – список \textbf{lst} – и возвращающую 1 значение: \textbf{True}, если все элементы списка чётные, и \textbf{False} в противном случае.
    \item Напишите функцию \textbf{square\_all}, принимающую 1 аргумент – список чисел \textbf{lst} – и возвращающую 1 значение: новый список с квадратами всех элементов.
    \item Напишите функцию \textbf{count\_long\_words}, принимающую 2 аргумента – строку \textbf{s} и число \textbf{n} – и возвращающую 1 значение: количество слов в строке длиной больше n.
    \item Напишите функцию \textbf{squares\_dict}, принимающую 1 аргумент – число \textbf{n} – и возвращающую 1 значение: словарь, где ключи – числа от 1 до n, а значения – их квадраты.
    \item Напишите функцию \textbf{count\_above\_average}, принимающую 1 аргумент – список чисел \textbf{lst} – и возвращающую 1 значение: количество чисел в списке, которые больше среднего значения списка.
    \item Напишите функцию \textbf{find\_all\_indices}, принимающую 2 аргумента – список \textbf{lst} и элемент \textbf{elem} – и возвращающую 1 значение: список всех индексов элемента \textbf{elem} в списке \textbf{lst}.
    \item Напишите функцию \textbf{flip\_signs}, принимающую 1 аргумент – список чисел \textbf{lst} – и возвращающую 1 значение: список с противоположными знаками элементов.
    \item Напишите функцию \textbf{acronym}, принимающую 1 аргумент – строку \textbf{s} – и возвращающую 1 значение: строку, состоящую из первых букв каждого слова в s.
    \item Напишите функцию \textbf{filter\_even\_length\_words}, принимающую 1 аргумент – строку \textbf{s} – и возвращающую 1 значение: список слов чётной длины.
    \item Напишите функцию \textbf{cube\_all}, принимающую 1 аргумент – список чисел \textbf{lst} – и возвращающую 1 значение: новый список с кубами элементов.
    \item Напишите функцию \textbf{is\_substring}, принимающую 2 аргумента – строки \textbf{s1} и \textbf{s2} – и возвращающую 1 значение: \textbf{True}, если \textbf{s1} является подстрокой \textbf{s2}, и \textbf{False} в противном случае.
    \item Напишите функцию \textbf{remove\_spaces}, принимающую 1 аргумент – строку \textbf{s} – и возвращающую 1 значение: строку без пробелов.
    \item Напишите функцию \textbf{is\_strictly\_increasing}, принимающую 1 аргумент – список lst – и возвращающую 1 значение: \textbf{True}, если список является строго возрастающей последовательностью, и \textbf{False} в противном случае.
    \item Напишите функцию \textbf{min\_max}, принимающую 1 аргумент – список чисел \textbf{lst} – и возвращающую 1 значение: кортеж с минимальным и максимальным элементами списка.
    \item Напишите функцию \textbf{alternate\_case}, принимающую 1 аргумент – строку \textbf{s} – и возвращающую 1 значение: строку с чередующимся регистром символов.
    \item Напишите функцию \textbf{flatten\_list}, принимающую 1 аргумент – список списков \textbf{lst} – и возвращающую 1 значение: плоский список.
    \item Напишите функцию \textbf{sum\_of\_digits}, принимающую 1 аргумент – число \textbf{n} – и возвращающую 1 значение: сумму его цифр.
    \item Напишите функцию \textbf{common\_elements}, принимающую 2 аргумента – списки \textbf{lst1} и \textbf{lst2} – и возвращающую 1 значение: список общих элементов.
    \item Напишите функцию \textbf{second\_smallest}, принимающую 1 аргумент – список чисел \textbf{lst} – и возвращающую 1 значение: второе по величине минимальное число.
    \item Напишите функцию \textbf{is\_pangram}, принимающую 1 аргумент – строку \textbf{s} – и возвращающую 1 значение: \textbf{True}, если строка содержит все буквы алфавита хотя бы один раз, и \textbf{False} в противном случае.
    \item Напишите функцию \textbf{reverse\_words}, принимающую 1 аргумент – строку \textbf{s} – и возвращающую 1 значение: строку с словами в обратном порядке.
    \item Напишите функцию \textbf{sum\_of\_cubes}, принимающую 1 аргумент – список чисел \textbf{lst} – и возвращающую 1 значение: сумму их кубов.
    \item Напишите функцию \textbf{replace\_vowels}, принимающую 2 аргумента – строку \textbf{s} и символ \textbf{ch} – и возвращающую 1 значение: строку с гласными, заменёнными на \textbf{ch}.
    \item Напишите функцию \textbf{has\_equal\_char\_frequency}, принимающую 1 аргумент – строку \textbf{s} – и возвращающую 1 значение: \textbf{True}, если строка содержит одинаковое количество каждой буквы, и \textbf{False} в противном случае.
    \item Напишите функцию \textbf{multiply\_list}, принимающую 1 аргумент – список чисел \textbf{lst} – и возвращающую 1 значение: произведение всех элементов списка.
    \item Напишите функцию \textbf{unique\_words}, принимающую 1 аргумент – список строк \textbf{lst} – и возвращающую 1 значение: строку, содержащую только уникальные слова.
    \item Напишите функцию \textbf{count\_vowels\_consonants}, принимающую 1 аргумент – строку \textbf{s} – и возвращающую 1 значение: кортеж с количеством гласных и согласных.
    %\item Напишите функцию \textbf{recursive\_factorial}, принимающую 1 аргумент – число n – и возвращающую 1 значение: факториал числа, вычисленный рекурсивно.
    %\item Напишите функцию \textbf{recursive\_fibonacci}, принимающую 1 аргумент – число n – и возвращающую 1 значение: n-ое число Фибоначчи, вычисленное рекурсивно.
    \item Напишите функцию \textbf{count\_unique}, принимающую 1 аргумент – список \textbf{lst} – и возвращающую 1 значение: количество уникальных чисел в списке.
    \item Напишите функцию \textbf{find\_pairs\_with\_sum}, принимающую 2 аргумента – список чисел \textbf{lst} и число \textbf{sum} – и возвращающую 1 значение: список пар чисел, сумма которых равна \textbf{sum}.
    \item Напишите функцию \textbf{can\_sort\_with\_one\_swap}, принимающую 1 аргумент – список чисел \textbf{lst} – и возвращающую 1 значение: \textbf{True}, если список можно упорядочить по возрастанию с одной заменой элемента, и \textbf{False} в противном случае.
    \item Напишите функцию \textbf{filter\_by\_length}, принимающую 2 аргумента – список строк \textbf{lst} и число \textbf{n} – и возвращающую 1 значение: список слов длиной больше или равной $n$.
    \item Напишите функцию \textbf{permutations}, принимающую 1 аргумент – строку s – и возвращающую 1 значение: список всех возможных перестановок символов строки.
    \item Напишите функцию \textbf{all\_sublists}, принимающую 1 аргумент – список lst – и возвращающую 1 значение: список всех возможных подсписков.
    \item Напишите функцию \textbf{word\_frequency}, принимающую 1 аргумент – строку s – и возвращающую 1 значение: словарь, где ключ – слово из строки, а значение – частота его появления.
    \item Напишите функцию \textbf{remove\_duplicates}, принимающую 1 аргумент – список lst – и возвращающую 1 значение: список без элементов, встречающихся более одного раза.
    \item Напишите функцию \textbf{to\_title\_case}, принимающую 1 аргумент – строку s – и возвращающую 1 значение: строку в формате заголовка.
    %\item Напишите функцию \textbf{recursive\_sum}, принимающую 1 аргумент – число \textbf{n} – и возвращающую 1 значение: сумму чисел от 1 до $n$, вычисленную рекурсивно.
    \item Напишите функцию \textbf{sentence\_count}, принимающую 1 аргумент – строку \textbf{s} – и возвращающую 1 значение: количество предложений в строке.
    \item Напишите функцию \textbf{all\_strings\_unique}, принимающую 1 аргумент – список строк \textbf{lst} – и возвращающую 1 значение: \textbf{True}, если все строки в списке уникальны, и \textbf{False} в противном случае.
    \item Напишите функцию \textbf{count\_less\_than}, принимающую 2 аргумента – список чисел \textbf{lst} и число \textbf{n} – и возвращающую 1 значение: количество чисел в списке, которые меньше $n$.
    %\item Напишите функцию \textbf{recursive\_max}, принимающую 1 аргумент – список чисел lst – и возвращающую 1 значение: наибольший элемент в списке, вычисленный рекурсивно.
    \item Напишите функцию \textbf{multiples}, принимающую 2 аргумента – числа \textbf{n} и \textbf{m} – и возвращающую 1 значение: список из первых $n$ чисел, кратных $m$.
    \item Напишите функцию \textbf{filter\_positive}, принимающую 1 аргумент – список чисел \textbf{lst} – и возвращающую 1 значение: список только с положительными числами.
    \item Напишите функцию \textbf{unique\_random\_numbers}, принимающую 2 аргумента – числа \textbf{n} и \textbf{range\_max} – и возвращающую 1 значение: список из $n$ уникальных случайных чисел в диапазоне от 0 до \textbf{range\_max}.
    \item Напишите функцию \textbf{capitalize\_alternate}, принимающую 1 аргумент – строку \textbf{s} – и возвращающую 1 значение: строку с каждой второй буквой заглавной.
    \item Напишите функцию \textbf{rotate\_list}, принимающую 2 аргумента – список \textbf{lst} и число \textbf{n} – и возвращающую 1 значение: список, циклически сдвинутый на $n$ позиций.
    \item Напишите функцию \textbf{divisors}, принимающую 1 аргумент – число \textbf{n} – и возвращающую 1 значение: список всех делителей числа $n$.
    % \item Напишите функцию \textbf{is\_armstrong}, принимающую 1 аргумент – число n – и возвращающую 1 значение: \textbf{True}, если число является числом Армстронга, и \textbf{False} в противном случае.
    \item Напишите функцию \textbf{sum\_of\_digits\_squared}, принимающую 1 аргумент – число \textbf{n} – и возвращающую 1 значение: сумму квадратов цифр числа.

    \item Напишите функцию \textbf{square\_properties}, принимающую 1 аргумент – сторону квадрата \textbf{a} – и возвращающую 3 значения (с помощью кортежа): периметр квадрата, площадь квадрата и диагональ квадрата.
    \item Напишите функцию \textbf{arithmetic\_progression}, принимающую 3 позиционных аргумента $a$, $n$ и $d$: первый $a$ - первый член арифметической прогрессии, второй $n$ - номер члена арифметической прогрессии ($n$ натуральное число), третий $d$ - разность арифметической прогрессии, - и возвращающую 1 значение: $n$-ый член арифметической прогрессии.
    \item Напишите функцию \textbf{geometric\_progression}, принимающую 3 позиционных аргумента $b$, $n$ и $q$: первый $b$ - первый член геометрической прогрессии, второй $n$ - номер члена геометрической прогрессии ($n$ натуральное число), третий $q$ - знаменатель геометрической прогрессии, - и возвращающую 1 значение: $n$-ый член геометрической прогрессии.
    \item Напишите функцию \textbf{perform\_operation}, принимающую 3 позиционных аргумента: первые два – числа \textbf{a} и \textbf{b}, третий – операция \textbf{op}, которая должна быть произведена над ними. Если третий аргумент `+`, сложить их; если `-`, то вычесть (из первого второе); `*` - умножить; `/` - разделить (первое на второе). В остальных случаях вернуть строку <<Неизвестная операция>>.
    \item Напишите функцию \textbf{compound\_interest}, принимающую 2 аргумента – сумму вклада \textbf{a} и срок вклада \textbf{years} – и возвращающую 1 значение: сумму, которая будет на счету пользователя через years лет при годовой процентной ставке 10\%.
\end{enumerate}


\textbf{Функции с именованными аргументами}

\begin{enumerate}
    \item Напишите функцию \textbf{arithmetic\_progression\_sum}, принимающую 3 именованных аргумента $a$, $n$ и $d$: первый $a$ - первый член арифметической прогрессии, второй $n$ - число первых членов арифметической прогрессии ($n$ натуральное число), третий $d$ - разность арифметической прогрессии, - и возвращающую 1 значение: сумму $n$ первых членов арифметической прогрессии. По умолчанию $d=1$.
    \item Напишите функцию \textbf{geometric\_progression\_sum}, принимающую 3 именованных аргумента $b$, $n$ и $q$: первый $b$ - первый член геометрической прогрессии, второй $n$ - число первых членов геометрической прогрессии ($n$ натуральное число), третий $q$ - знаменатель геометрической прогрессии, - и возвращающую 1 значение: сумму $n$ первых членов геометрической прогрессии. По умолчанию $q=1$.
    \item Напишите функцию \textbf{polynomial\_value}, принимающую 7 именованных аргументов: $x$, $a_0$, $a_1$, $a_2$, $a_3$, $a_4$ и $a_5$, - и возвращающую 1 значение: значение многочлена пятой степени $p(x)=a_0x^5+a_1x^4+a_2x^3+a_3x^2+a_4x+a_5$ с действительными коэффициентами. По умолчанию $a_i = 1$ для всех $i$.
    \item Напишите функцию \textbf{calculate\_bonus}, принимающую 4 аргумента: два именованных аргумента \textbf{name} и \textbf{last\_name} – имя и фамилию сотрудника, и два аргумента по умолчанию – \textbf{salary=120000} и \textbf{bonus=10} (оклад и премия), - и возвращающую 2 значения: размер новогодней премии для сотрудника и зарплату с учетом премии (см. примеры вызова).
    
    \begin{inputformat}[title=Примеры вызова функции]
        calculate\_bonus('Алина', 'Тимофеева', salary=150000, bonus=25) \\
        calculate\_bonus('Алексей', 'Ковалев', bonus=15) \\
        calculate\_bonus('Игорь', 'Ефимов') \\
        calculate\_bonus('Анастасия', 'Яковлева', salary=100000, bonus=20) 
    \end{inputformat}
            
    \begin{outputformat}[title=Вывод]
        Новогодняя премия сотрудника Алина Тимофеева: 37500.00 руб. \\
    Оклад: 150000.00 руб. \\
    Всего к выдаче: 187500.00 руб. \\

    Новогодняя премия сотрудника Алексей Ковалев: 18000.00 руб. \\
    Оклад: 120000.00 руб. \\
    Всего к выдаче: 138000.00 руб. \\

    Новогодняя премия сотрудника Игорь Ефимов: 12000.00 руб. \\
    Оклад: 120000.00 руб. \\
    Всего к выдаче: 132000.00 руб. \\

    Новогодняя премия сотрудника Анастасия Яковлева: 20000.00 руб. \\
    Оклад: 100000.00 руб. \\
    Всего к выдаче: 120000.00 руб. \\
    \end{outputformat}


    \item Напишите функцию \textbf{calculate\_discount}, принимающую 3 аргумента: два именованных аргумента \textbf{price} и \textbf{discount} – цена товара и размер скидки, и один аргумент по умолчанию \textbf{quantity=1} (количество товаров), - и возвращающую 1 значение: общую стоимость покупки с учетом скидки.
    \item Напишите функцию \textbf{calculate\_bmi}, принимающую 2 именованных аргумента – вес (\textbf{weight}) и рост (\textbf{height}) человека – и возвращающую 1 значение: индекс массы тела (\textbf{BMI}).
    \item Напишите функцию \textbf{calculate\_interest}, принимающую 3 именованных аргумента – сумму вклада (principal), процентную ставку (\textbf{rate}) и срок вклада в годах (\textbf{years}) – и возвращающую 1 значение: сумму процентов, начисленных за этот срок.
    \item Напишите функцию \textbf{calculate\_discount}, принимающую 2 именованных аргумента – цену товара (\textbf{price}) и процент скидки (\textbf{discount}) – и возвращающую 1 значение: цену товара после применения скидки.
    \item Напишите функцию \textbf{calculate\_future\_value}, принимающую 3 именованных аргумента – сумму вклада (\textbf{principal}), процентную ставку (\textbf{rate}) и срок вклада в годах (\textbf{years}) – и возвращающую 1 значение: будущую стоимость вклада.
    \item Напишите функцию \textbf{calculate\_age}, принимающую 1 именованных аргумент – дату рождения в формате 'YYYY-MM-DD' (\textbf{birthdate}) – и возвращающую 1 значение: возраст человека в годах.
    \item Напишите функцию \textbf{calculate\_days\_between\_dates}, принимающую 2 именованных аргумента – две даты (\textbf{date1} и \textbf{date2}) в формате 'YYYY-MM-DD' – и возвращающую 1 значение: количество дней между ними.
\end{enumerate}


\textbf{Функции с переменным числом аргументов}

\begin{enumerate}
    \item Напишите функцию \textbf{average\_numbers}, принимающую произвольное количество целых чисел (позиционных аргументов) и возвращающую 1 значение: среднее арифметическое без использования встроенных функций \texttt{sum()} и \texttt{len()}.
    \item Напишите функцию \textbf{sum\_of\_squares}, принимающую произвольное количество числовых аргументов и возвращающую 1 значение: сумму их квадратов.
    \item Напишите функцию \textbf{to\_uppercase\_words}, принимающую произвольное количество строк (позиционных аргументов) и возвращающую список строк, где каждое слово преобразовано в верхний регистр.
    \item Напишите функцию \textbf{concatenate\_strings}, принимающую произвольное количество строк (позиционных аргументов) и возвращающую 1 значение: объединённую строку.
    \item Напишите функцию \textbf{max\_number}, принимающую произвольное количество чисел (позиционных аргументов) и возвращающую 1 значение: наибольшее из них.
    \item Напишите функцию \textbf{min\_number}, принимающую произвольное количество чисел (позиционных аргументов) и возвращающую 1 значение: наименьшее из них.
    \item Напишите функцию \textbf{reverse\_strings}, принимающую произвольное количество строк (позиционных аргументов) и возвращающую список строк, где каждая строка перевёрнута.
    \item Напишите функцию \textbf{count\_vowels}, принимающую произвольное количество строк (позиционных аргументов) и возвращающую список, где каждый элемент представляет количество гласных в соответствующей строке.
    \item Напишите функцию \textbf{factorial\_numbers}, принимающую произвольное количество целых чисел (позиционных аргументов) и возвращающую список, где каждый элемент представляет факториал соответствующего числа.
    \item Напишите функцию \textbf{is\_palindrome}, принимающую произвольное количество строк (позиционных аргументов) и возвращающую список, где каждый элемент представляет \textbf{True}, если строка является палиндромом, и \textbf{False} в противном случае.
    \item Напишите функцию \textbf{remove\_spaces}, принимающую произвольное количество строк (позиционных аргументов) и возвращающую список строк без пробелов.
    \item Напишите функцию \textbf{sort\_numbers}, принимающую произвольное количество чисел (позиционных аргументов) и возвращающую список, отсортированный по возрастанию.
    \item Напишите функцию \textbf{unique\_elements}, принимающую произвольное количество аргументов и возвращающую список уникальных элементов.
    \item Напишите функцию \textbf{count\_characters}, принимающую произвольное количество строк (позиционных аргументов) и возвращающую список, где каждый элемент представляет количество символов в соответствующей строке.
    \item Напишите функцию \textbf{multiply\_numbers}, принимающую произвольное количество чисел (позиционных аргументов) и возвращающую 1 значение: произведение всех чисел.
    \item Напишите функцию \textbf{is\_prime}, принимающую произвольное количество целых чисел (позиционных аргументов) и возвращающую список, где каждый элемент представляет \textbf{True}, если число простое, и \textbf{False} в противном случае.
    \item Напишите функцию \textbf{join\_with\_hyphen}, принимающую произвольное количество строк (позиционных аргументов) и возвращающую 1 значение: строку, где все элементы соединены дефисом.
    \item Напишите функцию \textbf{cube\_numbers}, принимающую произвольное количество чисел (позиционных аргументов) и возвращающую список, где каждый элемент представляет куб соответствующего числа.
    \item Напишите функцию \textbf{reverse\_words}, принимающую произвольное количество строк (позиционных аргументов) и возвращающую список строк, где слова в каждой строке расположены в обратном порядке.
\end{enumerate}


\end{document}
