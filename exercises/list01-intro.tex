\documentclass[a4,12pt]{article}
\usepackage[utf8]{inputenc}
\usepackage[T2A]{fontenc}
\usepackage[english, russian]{babel}



\usepackage{tcolorbox}
% Определяем окружение для Формата ввода
\newtcolorbox{inputformat}[1][]{colback=blue!5!white, colframe=blue!65!black,
    fonttitle=\bfseries, title=Формат ввода, #1}

% Определяем окружение для Формата вывода
\newtcolorbox{outputformat}[1][]{colback=green!5!white, colframe=green!65!black,
    fonttitle=\bfseries, title=Формат вывода, #1}

% Определяем окружение для комментария
\newtcolorbox{exercisecomment}[1][]{colback=yellow!5!white, colframe=yellow!80!black, fonttitle=\bfseries, title=Комментарий, #1}

% Определяем окружение для замечания
\newtcolorbox{exercisenote}[1][]{colback=red!5!white, colframe=red!80!black, fonttitle=\bfseries, title=Замечание, #1}


\usepackage{geometry}
%\geometry{papersize={20.3 cm,25.4 cm}}
\geometry{left=2cm}
\geometry{right=2cm}
\geometry{top=2cm}
\geometry{bottom=2cm}


\usepackage{amsmath, amsfonts, amsthm, amssymb, amsopn, amscd}
\usepackage{enumerate}
\usepackage{enumitem}
\usepackage[mathscr]{eucal}

\usepackage{hyperref}
\hypersetup{unicode=true,final=true,colorlinks=true}

\theoremstyle{remark}
%\newtheorem{exercise}{Упражнение}
\newtheorem{exercise}{\textbf{Упражнение}}[section]
\renewcommand{\theexercise}{\textbf{\arabic{exercise}}}
%\renewcommand{\theexercise}{\textbf{\#\arabic{exercise}}}


\title{Листок 01. Ввод, вывод данных и переменные в Python}

\author{А.Н. Лата}

\begin{document}

%\maketitle

%\tableofcontents

\section*{\centering Листок 01. Ввод, вывод данных и переменные в Python}

\begin{exercisenote}[title=Замечания]
\begin{itemize}
    \item для решение приведенных ниже упражнений не требуется создавать (определять) пользовательские функции.
    \item при решении упражнений полученные результаты выведите на экран с помощью функции {\color{blue}{print()}}.
    \item позаботесь о том, чтобы выводимый на экран результат был снабжен информацией о нем, там где это необходимо.
    \item не забывайте писать комментарии к вашему коду
\end{itemize}
\end{exercisenote}


\begin{exercise}[Приветствие пользователя]
%Запросите у пользователя его
Напишите программу, которая запрашивает имя и возраст пользователя, и выводит приветствие и возраст пользователя в следующем году
\begin{enumerate}[label=\textbf{\alph*)}]
    \item без использования f-строки
    \item с использованием f-строки
\end{enumerate}

\begin{inputformat}
Катя\\
14
\end{inputformat}
    
\begin{outputformat}
Привет, Катя!\\
В следующем году тебе будет 15!
\end{outputformat}

\begin{exercisecomment}
Фраза \textbf{<<Напишите программу, которая запрашивает>>} равносильна фразе \textbf{<<Запросите у пользователя>>}.
\end{exercisecomment}

\end{exercise}

\begin{exercise}
Напишите программу, которая запрашивает имя, фамилию и возраст пользователя, а затем выводит эти данные в столбик % с помощью $f$-строки.
\begin{enumerate}[label=\textbf{\alph*)}]
    \item без использования f-строки
    \item с использованием f-строки
\end{enumerate}

\begin{inputformat}
Евгения\\
Фролова\\
19
\end{inputformat}
    
\begin{outputformat}
Имя: Евгения\\
Фамилия: Фролова\\
Возраст: 19
\end{outputformat}

\end{exercise}

\begin{exercise}[Сложение чисел]
Запросите у пользователя два числа и выведите их сумму
\begin{enumerate}[label=\textbf{\alph*)}]
    \item без использования f-строки
    \item с использованием f-строки
\end{enumerate}
\begin{inputformat}
3\\
5
\end{inputformat}
    
\begin{outputformat}
Сумма чисел 3 и 5 равна 8.
\end{outputformat}
\end{exercise}

\begin{exercise}
%Напишите программу, которая получает на вход 
Запросите у пользователя целое число $n$ из диапазона от 1 до 9 включительно, и выведите результат вычисления выражения $nnn - nn - n$. %Например, если введенное число 3, программа должна вывести 297 (333 – 33 – 3).

\begin{inputformat}
Введенно число 3
\end{inputformat}
    
\begin{outputformat}
Программа должна вывести 297 (так как 333 – 33 – 3)
\end{outputformat}

\end{exercise}

\begin{exercise}[Периметр прямоугольника]
Запросите у пользователя длину и ширину прямоугольника и выведите его периметр
\begin{enumerate}[label=\textbf{\alph*)}]
    \item без использования f-строки
    \item с использованием f-строки
\end{enumerate}
\begin{inputformat}
3\\
4
\end{inputformat}
    
\begin{outputformat}
Периметр прямоугольника: 14
\end{outputformat}
\end{exercise}

\begin{exercise}[Остаток от деления]
Запросите у пользователя два числа и выведите остаток от их деления.

\begin{inputformat}
5\\
3
\end{inputformat}

\begin{outputformat}
Остаток от деления 5 на 3 равен 2
\end{outputformat}
\end{exercise}

\begin{exercise}[Округление числа]
Запросите у пользователя вещественное число и выведите его, округлённое до двух знаков после запятой
\begin{enumerate}[label=\textbf{\alph*)}]
    \item без использования f-строки
    \item с использованием f-строки
\end{enumerate}
\begin{inputformat}
6.7896
\end{inputformat}

\begin{outputformat}
Число 6.7896 округлённое до двух знаков: 6.79
\end{outputformat}
\end{exercise}

\begin{exercise}[Температура в Фаренгейтах]
Запросите у пользователя температуру в Цельсиях и переведите её в Фаренгейты по формуле: $F=\dfrac{9}{5}C + 32$ с точностью до 2-х десятичных знаков.

\begin{inputformat}
0
\end{inputformat}

\begin{outputformat}
Температура в Фаренгейтах: 32
\end{outputformat}
\end{exercise}

\begin{exercise}[Среднее арифметическое]
Запросите у пользователя три числа и выведите их среднее арифметическое с точностью до 2-х десятичных знаков.

\begin{inputformat}
2\\
-3\\
7
\end{inputformat}

\begin{outputformat}
Среднее арифметическое чисел 2, -3 и 7 равно 2
\end{outputformat}
\end{exercise}

\begin{exercise}[Минимум и максимум]
Запросите у пользователя три числа и выведите минимальное и максимальное из них.

\begin{inputformat}
3\\
-1\\
5
\end{inputformat}

\begin{outputformat}
Минимальное число: -1, максимальное число: 5
\end{outputformat}
\end{exercise}

\begin{exercise}[Возведение в степень]
Запросите у пользователя число и степень, затем выведите результат возведения числа в эту степень. 

\begin{inputformat}
2\\
3
\end{inputformat}

\begin{outputformat}
2 в степени 3 равно 8
\end{outputformat}
\end{exercise}

\begin{exercise}[Длина гипотенузы]
Запросите у пользователя длины катетов прямоугольного треугольника и выведите длину гипотенузы, используя теорему Пифагора, с точностью до 2-х десятичных знаков.

\begin{inputformat}
3\\
4
\end{inputformat}

\begin{outputformat}
Гипотенуза треугольника с катетами 3 и 4 равна 5
\end{outputformat}
\end{exercise}

\end{document}