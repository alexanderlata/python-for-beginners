\documentclass[a4,12pt]{article}
\usepackage[utf8]{inputenc}
\usepackage[T2A]{fontenc}
\usepackage[english, russian]{babel}


\usepackage{csquotes}
\usepackage{epigraph}
\usepackage{tcolorbox}
% Определяем окружение для Формата ввода
\newtcolorbox{inputformat}[1][]{colback=blue!5!white, colframe=blue!65!black,
    fonttitle=\bfseries, title=Формат ввода, #1}

% Определяем окружение для Формата вывода
\newtcolorbox{outputformat}[1][]{colback=green!5!white, colframe=green!65!black,
    fonttitle=\bfseries, title=Формат вывода, #1}

% Определяем окружение для комментария
\newtcolorbox{exercisecomment}[1][]{colback=yellow!5!white, colframe=yellow!80!black, fonttitle=\bfseries, title=Комментарий, #1}

% Определяем окружение для замечания
\newtcolorbox{exercisenote}[1][]{colback=red!5!white, colframe=red!80!black, fonttitle=\bfseries, title=Замечание, #1}


\usepackage{geometry}
%\geometry{papersize={20.3 cm,25.4 cm}}
\geometry{left=2cm}
\geometry{right=2cm}
\geometry{top=2cm}
\geometry{bottom=2cm}


\usepackage{amsmath, amsfonts, amsthm, amssymb, amsopn, amscd}
\usepackage{enumerate}
\usepackage{enumitem}
\usepackage[mathscr]{eucal}

\usepackage{hyperref}
\hypersetup{unicode=true,final=true,colorlinks=true}

\theoremstyle{remark}
%\newtheorem{exercises}{Упражнения}
\newtheorem{exercise}{\textbf{Упражнение}}[section]
\renewcommand{\theexercise}{\textbf{\arabic{exercise}}}
%\renewcommand{\theexercise}{\textbf{\#\arabic{exercise}}}


\title{Листок 02. Методы работы со строками в Python}

\author{А.Н. Лата}

\begin{document} 

%\maketitle

\section*{\centering Листок 02. Методы работы со строками в Python}

\begin{exercisenote}[title=Замечания]
\begin{itemize}
    \item для решение приведенных ниже упражнений не требуется создавать (определять) пользовательские функции.
    \item при решении упражнений полученные результаты выведите на экран с помощью функции {\color{blue}{print()}}.
    \item позаботесь о том, чтобы выводимый на экран результат был снабжен информацией о нем, там где это необходимо.
    \item не забывайте писать комментарии к вашему коду
\end{itemize}
\end{exercisenote}

%\tableofcontents
%\section{\centering Ввод и вывод данных}

\textbf{Упражнения}
\begin{enumerate}
    \item Преобразуйте строку "python" к заглавным буквам.
    \item Преобразуйте строку "HELLO" в строчные буквы.
    \item Сделайте первую букву строки "hello world" заглавной.
    \item Сделайте заглавными первые буквы каждого слова в строке "hello world python".
    \item Удалите лишние пробелы в начале и в конце строки " hello world ".
    \item Замените слово "world" на "Python" в строке "Hello world".
    \item Найдите индекс первого вхождения подстроки "Python" в строке "I love Python".
    \item Подсчитайте количество вхождений символа "a" в строке "banana".
    \item Разделите строку "apple, orange, banana" на список фруктов.
    \item Соедините элементы списка ['Hello', 'Python'] в строку через пробел.
    \item Проверьте, начинается ли строка "Hello world" со слова "Hello".
    \item Проверьте, заканчивается ли строка "Hello world" словом "Python".
    \item Проверьте, состоит ли строка "abcDEF" только из букв.
    \item Проверьте, состоит ли строка "123456" только из цифр.
    \item Проверьте, состоит ли строка "Data Science" только из пробелов.
    \item Центрируйте строку "Hi" до длины 10, используя символ *.
    \item Дополните строку "42" нулями слева до длины 6.
    \item Поменяйте регистр символов строки "Hello World" на противоположный.
    \item Разделите строку "python programming" на список слов.
    \item Замените все пробелы в строке "good bye world" на символ подчеркивания.
    \item Преобразуйте строку "Data Science" в верхний регистр.
    \item Преобразуйте строку "PYTHON" в нижний регистр.
    \item Сделайте первую букву строки "machine learning" заглавной.
    \item Сделайте заглавными первые буквы каждого слова в строке "data analysis and visualization".
    \item Удалите пробелы только слева в строке " leading spaces".
    \item Удалите пробелы только справа в строке "trailing spaces ".
    \item Замените все буквы "a" на "o" в строке "banana".
    \item Найдите индекс первого вхождения подстроки "learn" в строке "I want to learn Python".
    \item Подсчитайте количество вхождений подстроки "ing" в строке "learning is fun and interesting".
    \item Разделите строку "red;green;blue" на список цветов по разделителю ";".
    \item Соедините элементы списка ['Data', 'Science', 'AI'] в строку через дефис (-).
    \item Проверьте, начинается ли строка "Artificial Intelligence" со слова "Artificial".
    \item Проверьте, заканчивается ли строка "Deep Learning" словом "Python".
    \item Проверьте, состоит ли строка "OpenAI" только из букв.
    \item Проверьте, состоит ли строка "2024" только из цифр.
    \item Проверьте, состоит ли строка "\textbackslash t\textbackslash n " только из пробельных символов.
    \item Центрируйте строку "Center" до длины 15, используя символ -.
    \item Дополните строку "7" нулями слева до длины 4.
    \item Поменяйте регистр символов строки "Python Is Fun" на противоположный.
    \item Разделите строку "one two three four" на список слов.
    \item Замените все пробелы в строке "hello world" на символ "*".
    \item Преобразуйте строку "MixedCase" в полностью заглавные буквы.
    \item Преобразуйте строку "MIXEDcase" в полностью строчные буквы.
    \item Сделайте первую букву каждого слова в строке "welcome to python programming" заглавной.
    \item Удалите символы "!" с начала и конца строки "!!!Hello!!!".
    \item Замените первую букву "h" на "H" в строке "hello".
    \item Найдите индекс подстроки "Py" в строке "Python Programming".
    \item Подсчитайте количество пробелов в строке "count the spaces here".
    \item Разделите строку "apple|banana|cherry" на список фруктов по разделителю "|".
    \item Соедините элементы списка ['2024', 'AI', 'ML'] в строку через пробел.
    \item Проверьте, начинается ли строка "Hello, Python" со строки "Hi".
    \item Проверьте, заканчивается ли строка "Learning Python" словом "Python".
    \item Проверьте, состоит ли строка "Python3" только из букв.
    \item Проверьте, состоит ли строка "123abc" только из цифр.
    \item Проверьте, состоит ли строка " " только из пробелов.
    \item Центрируйте строку "Align" до длины 12, используя символ ".".
    \item Дополните строку "99" нулями слева до длины 5.
    \item Поменяйте регистр символов строки "Good Morning" на противоположный.
    \item Разделите строку "split,this,string" на список слов по разделителю ",".
    \item Замените все символы "o" в строке "foo bar" на "0".
    \item Преобразуйте строку "Convert Me" в нижний регистр и затем в верхний.
    \item Сделайте первую букву строки "data" заглавной, а остальные строчными.
    \item Замените все вхождения "e" на "3" в строке "test example".
    \item Найдите индекс первого вхождения буквы "a" в строке "alphabet".
    \item Подсчитайте количество букв "t" в строке "testing the method".
    \item Разделите строку "2024/09/18" на список дат по разделителю "/".
    %\item Соедините элементы списка ['year', 'month', 'day'] в строку через символ "-".
    \item Проверьте, начинается ли строка "Start Here" со слова "Start".
    \item Проверьте, заканчивается ли строка "End Game" словом "Game".
    \item Проверьте, состоит ли строка "OnlyLetters" только из букв.
    \item Проверьте, состоит ли строка "NoDigits123" только из цифр.
    \item Проверьте, состоит ли строка "\textbackslash n\textbackslash t " только из пробельных символов.
    \item Центрируйте строку "Middle" до длины 20, используя символ "*".
    \item Дополните строку "5" нулями слева до длины 3.
    \item Поменяйте регистр символов строки "Case Swap" на противоположный.
    \item Разделите строку "alpha,beta,gamma,delta" на список слов по разделителю ",".
    \item Замените все пробелы в строке "space here" на "+".
    \item Преобразуйте строку "Example" в заглавные буквы и затем обратно в строчные.
    \item Сделайте первую букву каждого слова в строке "learn python programming" заглавной.
    \item Удалите все символы "*" с начала и конца строки "***star***".
    \item Замените первую букву "s" на "S" в строке "start".
    \item Найдите индекс подстроки "gram" в строке "programming".
    \item Подсчитайте количество букв "m" в строке "programming is fun".
    \item Разделите строку "key1:value1;key2:value2" на список по разделителю ";".
    \item Соедините элементы списка ['key1=value1', 'key2=value2'] в строку через ",".
    \item Проверьте, начинается ли строка "Function test" со слова "Test".
    \item Проверьте, заканчивается ли строка "Final Test" словом "Test".
    \item Проверьте, состоит ли строка "Test123" только из букв.
    \item Проверьте, состоит ли строка "456789" только из цифр.
    \item Проверьте, состоит ли строка "  \textbackslash t\textbackslash n " только из пробелов.
    \item Центрируйте строку "Edge" до длины 8, используя символ "+".
    \item Дополните строку "8" нулями слева до длины 2.
    \item Поменяйте регистр символов строки "Flip CASE" на противоположный.
    \item Разделите строку "one|two|three|four" на список слов по разделителю "|".
    \item Замените все буквы "e" в строке "Elephant" на "3".
    \item Преобразуйте строку "convert me" в заглавные буквы и затем в нижние.
    \item Сделайте первую букву строки "python" заглавной, а остальные оставьте без изменений.
    \item Замените все вхождения "o" на "0" в строке "look for opportunities".
    \item Найдите индекс первого вхождения подстроки "for" в строке "search for a keyword".
    \item Подсчитайте количество вхождений подстроки "or" в строке "for or against".
\end{enumerate}



\end{document}