\documentclass[a4,12pt]{article}
\usepackage[utf8]{inputenc}
\usepackage[T2A]{fontenc}
\usepackage[english, russian]{babel}


\usepackage{csquotes}
\usepackage{epigraph}
\usepackage{tcolorbox}
% Определяем окружение для Формата ввода
\newtcolorbox{inputformat}[1][]{colback=blue!5!white, colframe=blue!65!black,
    fonttitle=\bfseries, title=Формат ввода, #1}

% Определяем окружение для Формата вывода
\newtcolorbox{outputformat}[1][]{colback=green!5!white, colframe=green!65!black,
    fonttitle=\bfseries, title=Формат вывода, #1}

% Определяем окружение для комментария
\newtcolorbox{exercisecomment}[1][]{colback=yellow!5!white, colframe=yellow!80!black, fonttitle=\bfseries, title=Комментарий, #1}

% Определяем окружение для замечания
\newtcolorbox{exercisenote}[1][]{colback=red!5!white, colframe=red!80!black, fonttitle=\bfseries, title=Замечание, #1}


\usepackage{geometry}
%\geometry{papersize={20.3 cm,25.4 cm}}
\geometry{left=2cm}
\geometry{right=2cm}
\geometry{top=2cm}
\geometry{bottom=2cm}


\usepackage{amsmath, amsfonts, amsthm, amssymb, amsopn, amscd}
\usepackage{enumerate}
\usepackage{enumitem}
\usepackage[mathscr]{eucal}

\usepackage{hyperref}
\hypersetup{unicode=true,final=true,colorlinks=true}

\theoremstyle{remark}
%\newtheorem{exercises}{Упражнения}
\newtheorem{exercise}{\textbf{Упражнение}}[section]
\renewcommand{\theexercise}{\textbf{\arabic{exercise}}}
%\renewcommand{\theexercise}{\textbf{\#\arabic{exercise}}}


\title{Листок 06. Циклы в Python}

\author{А.Н. Лата}

\begin{document} 

%\maketitle

\section*{\centering Листок 06. Циклы в Python}

\begin{exercisenote}[title=Замечания]
\begin{itemize}
    \item для решение приведенных ниже упражнений не требуется создавать (определять) пользовательские функции.
    \item при решении упражнений полученные результаты выведите на экран с помощью функции {\color{blue}{print()}}.
    \item позаботесь о том, чтобы выводимый на экран результат был снабжен информацией о нем, там где это необходимо.
    \item не забывайте писать комментарии к вашему коду
\end{itemize}
\end{exercisenote}

%Выведите означает выведите на экран

\centering{\textbf{\Large Упражнения}}
\begin{enumerate}
    \subsection*{Начальный уровень}%(Базовые задачи)
    \item Выведите числа от 1 до 10.
    \item Выведите все чётные числа от 2 до 20.
    \item Выведите все числа от 1 до 10 в обратном порядке.
    \item Выведите сумму чисел от 1 до 100.
    \item Выведите сумму всех нечётных чисел от 1 до 50.
    \item Выведите произведение чисел от 1 до 10.
    \item Выведите квадраты чисел от 1 до 10.
    \item Выведите таблицу умножения на 5.
    \item Выведите все буквы строки 'Python' по одной в строке.
    \item Подсчитайте количество букв 'а' в строке 'Ананас' используя циклы.
    \item Подсчитайте количество цифр в строке 'Python 3.9'.
    \item Выведите символы строки 'Python' в обратном порядке.
    \item Создайте список, содержащий квадраты чисел от 1 до 10.
    
    %\item Выведите на экран число, которое является суммой квадратов первых 10 натуральных чисел.
    \subsection*{Средний уровень}
    \item Выведите все числа от 1 до 100, которые делятся на 7.
    \item Выведите все числа от 1 до 100, пропуская числа, которые делятся на 5.
    \item Найдите все числа от 1 до 50, которые делятся на 3 и 5.
    \item Создайте список из всех делителей числа 36.
    \item Создайте список с квадратами нечётных чисел от 1 до 20.
    \item Найдите сумму квадратов чисел от 1 до 20.
    \item Найдите сумму первых 10 натуральных чисел, кратных 3.
    \item Найдите все двузначные числа, сумма цифр которых равна 8.
    \item Подсчитайте количество гласных и согласных в строке.
    \item Найдите наибольший общий делитель двух чисел.
    \item Выведите первые $10$ чисел последовательности Фибоначчи.
    \item Проверьте, является ли число палиндромом.
    \item Подсчитайте сумму цифр числа 12345.
    \item Выведите каждый второй символ строки 'abcdef'.
    \item Найдите сумму всех элементов списка [1, 2, 3, 4, 5] не используя функцию $sum()$.
    \item Создайте новый список, содержащий только чётные числа из списка [1, 2, 3, 4, 5, 6, 7, 8, 9, 10].
    \item Вычислите произведение всех чисел в списке [1, 3, 5, 7, 9].
    \item Вычислите сумму всех чисел в двумерном списке [[1, 2], [3, 4], [5, 6]].
    \item Найдите сумму всех нечётных чисел в строке '1a2b3c4d5e6f7g8h9i'.
    \item Найдите факториал числа $10$.
    \item Вывесдите все гласные в строке 'Hello World'.
    \item Подсчитайте, сколько раз встречается слово 'apple' в списке ['apple', 'banana', 'apple', 'orange'].
    \item Создайте список, содержащий строки, длина которых больше 3 в списке ['cat', 'elephant', 'dog', 'mouse'].
    \item Выведите первый символ каждого слова в строке 'Hello World from Python'.
    \item Составьте строку, состоящую из первых букв каждого слова в строке 'Python is great'.
    \item Создайте строку, состоящую из первых трех букв каждого слова в строке 'I love programming'.
    \item Сформируйте список, содержащий только те элементы, которые начинаются на букву 'a' в строке 'apple, banana, avocado'.
    
    \subsection*{Продвинутый уровень}
    \item Выведите все элементы списка [1, 5, 3, 5, 7, 5, 9, 5, 5], за исключением числа 5.
    \item Найдите минимальное и максимальное число в списке [3, 5, 1, 8, 9, 2] не используя функции $min()$ и $max()$.
    \item Посчитайте количество чисел в списке [1, 6, 3, 7, 8, 2, 10], которые больше 5.
    \item Найдите самую длинную последовательность одинаковых символов в строке 'aabbbccccdddd'.
    \item Найти наибольшее число в списке [2, 3, 5, 7, 8, 10, 13], которое меньше 10.
    
    \item Создайте список, содержащий все буквы из строки 'python programming', которые встречаются больше одного раза.
    \item Найдите все простые числа до заданного числа (решето Эратосфена).
    \item Найдите все совершенные числа до 1000 (Число называется совершенным если оно равно сумме своих делителей).
    \item Выведите числа Армстронга до 1000 (сумма цифр в степени количества цифр равна самому числу).
    \item Найдите все числа Харшад (делятся на сумму своих цифр) до 1000.
    \item Проверьте, является ли число автоморфным (квадрат числа оканчивается самим числом).
    \item Найдите все числа-палиндромы, которые при возведении в квадрат дают палиндром.

\end{enumerate}

\newpage

\section*{\centering Задачи на списковые включения}

\begin{exercisenote}[title=Замечания]
    \begin{itemize}
        \item для решение приведенных ниже упражнений не требуется создавать (определять) пользовательские функции.
        \item для решения приведенных ниже упражнений используйте списковые включения.
    \end{itemize}
\end{exercisenote}


\begin{enumerate}
    \subsection*{Начальный уровень}
    \item Создайте список квадратов чисел от 1 до 10.
    \item Создайте список только четных чисел от 0 до 20.
    \item Дан список строк ['hello', 'python', 'code']. Создайте новый список, где каждая строка продублирована (например, из 'hello' получается 'hellohello').
    \item Дан список слов ['python', 'java', 'javascript', 'css']. Создайте список, содержащий длину каждого слова.
    \item Дано слово 'python'. Создайте список, состоящий из букв этого слова в верхнем регистре.
    
    \subsection*{Средний уровень}
    \item Создайте список кортежей, где каждый кортеж содержит число и его квадрат для чисел от 1 до 5.
    \item Дан список строк ['cat', 'dog', 'elephant', 'bee', 'butterfly']. Создайте новый список, содержащий только те строки, длина которых больше 3 символов.
    \item Создайте список всех чисел от 1 до 30, которые делятся либо на 3, либо на 5.
    \item Дан список [0, 10, 20, 25, 30] температур по Цельсию. Преобразуйте их в температуры по Фаренгейту.
    \item Дано предложение 'Python is a great programming language'. Создайте список из первых букв каждого слова (только для непустых слов).

    \subsection*{Продвинутый уровень}
    \item Дана строка 'python'. Создайте список всех возможных подстрок длины 2.
    \item Дан список дат ['2024-03-01', '2024-03-15', '2024-04-01'] в формате 'YYYY-MM-DD'. Преобразуйте их в список кортежей (год, месяц, день).
    \item Дано предложение 'Python programming is very interesting'. Создайте список слов, которые содержат хотя бы одну гласную букву.
    \item Найдите все числа-палиндромы в диапазоне от 100 до 1000.
    \item Дан список чисел [1, 2, 3, 4, 5]. Создайте список, содержащий суммы каждой пары соседних элементов.
    \item Дан список строк ['contact@example.com', 'python.org', 'user@mail.com', 'no-reply@site.com', 'invalid.email']. Извлеките из него все email адреса (строки, содержащие символ '@').
    \item Создайте список всех дат определенного месяца (например, март 2024) в формате 'YYYY-MM-DD'.
    \item Среди чисел от 1 до 20 найдите все тройки Пифагора (a, b, c), для которых выполняется равенство $a^2 + b^2 = c^2$.

\end{enumerate}

\end{document}