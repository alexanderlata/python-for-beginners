\documentclass[a4,12pt]{article}
\usepackage[utf8]{inputenc}
\usepackage[T2A]{fontenc}
\usepackage[english, russian]{babel}


\usepackage{csquotes}
\usepackage{epigraph}
\usepackage{tcolorbox}
% Определяем окружение для Формата ввода
\newtcolorbox{inputformat}[1][]{colback=blue!5!white, colframe=blue!65!black,
    fonttitle=\bfseries, title=Формат ввода, #1}

% Определяем окружение для Формата вывода
\newtcolorbox{outputformat}[1][]{colback=green!5!white, colframe=green!65!black,
    fonttitle=\bfseries, title=Формат вывода, #1}

% Определяем окружение для комментария
\newtcolorbox{exercisecomment}[1][]{colback=yellow!5!white, colframe=yellow!80!black, fonttitle=\bfseries, title=Комментарий, #1}

% Определяем окружение для замечания
\newtcolorbox{exercisenote}[1][]{colback=red!5!white, colframe=red!80!black, fonttitle=\bfseries, title=Замечание, #1}


\usepackage{geometry}
%\geometry{papersize={20.3 cm,25.4 cm}}
\geometry{left=2cm}
\geometry{right=2cm}
\geometry{top=2cm}
\geometry{bottom=2cm}


\usepackage{amsmath, amsfonts, amsthm, amssymb, amsopn, amscd}
\usepackage{enumerate}
\usepackage{enumitem}
\usepackage[mathscr]{eucal}

\usepackage{hyperref}
\hypersetup{unicode=true,final=true,colorlinks=true}

\theoremstyle{remark}
%\newtheorem{exercises}{Упражнения}
\newtheorem{exercise}{\textbf{Упражнение}}[section]
\renewcommand{\theexercise}{\textbf{\arabic{exercise}}}
%\renewcommand{\theexercise}{\textbf{\#\arabic{exercise}}}


\title{Листок 05. Условный оператор в Python}

\author{А.Н. Лата}

\begin{document} 

%\maketitle

\section*{\centering Листок 05. Условный оператор в Python}

\begin{exercisenote}[title=Замечания]
\begin{itemize}
    \item для решение приведенных ниже упражнений не требуется создавать (определять) пользовательские функции.
    \item при решении упражнений полученные результаты выведите на экран с помощью функции {\color{blue}{print()}}.
    \item позаботесь о том, чтобы выводимый на экран результат был снабжен информацией о нем, там где это необходимо.
    \item не забывайте писать комментарии к вашему коду
\end{itemize}
\end{exercisenote}

\textbf{Упражнения}
\begin{enumerate}
    \item Дано число. Проверьте, положительное оно, отрицательное или ноль.
    \item Дано число. Проверьте, является ли число чётным. \textit{Если число четное, вывести 'Четное', иначе — 'Нечетное'.}
    \item Даны два числа. Вывести большее из них. \textit{Если два числа равны, вывести 'Числа равны', иначе — вывести большее из них.}
    \item Дано целое число. Определите, является ли число трехзначным.
    \item Дано число. Проверьте, кратно ли число 7.
    \item Дано число. Определите, делится ли число на 2 и не делится на 3 одновременно.
    \item Даны три числа. Определите знак произведения трех чисел.
    \item Дано число. Проверьте, лежит ли число в интервале от -5 до 3. \textit{Если число больше -5 и меньше 3, вывести 'Число в диапазоне', иначе — 'Число вне диапазона'.}
    \item Дано целое число. Определите, является ли число отрицательным и четным одновременно. \textit{Если число отрицательное и четное, вывести 'Отрицательное четное', если отрицательное и нечетное — 'Отрицательное нечетное', если неотрицательное — 'Неотрицательное'.}
    \item Дана строка. Проверьте, начинается ли строка с русской буквы 'A'.
    \item Дана строка. Проверьте, является ли первая буква строки заглавной.
    \item Дана строка. Определите, оканчивается ли строка на точку.
    \item Дана строка. Проверьте, является ли строка пустой.
    \item Дана строка. Проверьте, содержит ли строка пробелы.
    \item Дана строка. Проверьте, состоит ли строка только из цифр.
    \item Дана строка. Проверьте, является ли строка палиндромом. \textit{Если строка читается одинаково слева направо и справа налево, вывести 'Палиндром', иначе — 'Не палиндром'.}
    \item Дан список. Определите, равны ли первый и последний элементы списка.
    \item Дан список. Определите, есть ли в списке хотя бы один элемент.
    \item Дан кортеж, содержащий числа. Проверьте, содержит ли кортеж значение 7. \textit{Вывести 'Семерка найдена', иначе — 'Семерка отсутствует'.}
    \item Дан список из пяти чисел. Определите, является ли элемент с индексом 2 четным числом. \textit{Если элемент на индексе 2 в списке четный, вывести 'Четное', иначе — 'Нечетное'.}
    \item Определите, является ли прямоугольник квадратом.
    \item Определите, лежит ли точка с координатами $(x, y)$ в первой четверти координатной плоскости.
    \item Проверьте, может ли существовать треугольник с заданными сторонами.
    \item Проверьте, лежит ли точка внутри круга радиуса $R$ с центром в начале координат.
    \item Проверьте, является ли введенный пользователем символ гласной буквой (для русского алфавита).
    \item Проверить, состоит ли строка только из заглавных букв.
    \item Определите, лежит ли точка внутри прямоугольника с заданными координатами углов.
    \item Проверите, является ли треугольник прямоугольным.
    \item Проверите, образуют ли три числа арифметическую прогрессию.
    \item Проверите, является ли число автоморфным (его квадрат оканчивается этим же числом).
    \item Проверите, все ли цифры числа, введенного пользователем, различны. \\ \textbf{Подсказка:} используйте функцию $set()$.
    \item Проверите, является ли сумма первой и последней цифры числа, введенного пользователем, четной.
    \item Дано положительное четырехзначное число. Проверите, равна ли сумма первой и последней цифры произведению средних цифр.
\end{enumerate}

\section*{\centering Усложненные задачи}
    \textit{Для каждой задачи нужно использовать вложенные условные операторы if-else для обработки всех возможных комбинаций входных данных. При решении важно учесть все граничные случаи и обработать некорректные входные данные.}

\begin{enumerate}
    \setcounter{enumi}{33}
    \item Напишите программу, которая принимает возраст человека и определяет, к какой возрастной категории он относится:
    \begin{itemize}
    \item 0-2: младенец
    \item 3-6: дошкольник
    \item 7-17: школьник
    \item 18-22: студент
    \item 23-60: работник
    \item 61+: пенсионер
    \end{itemize}
    
    При этом возраст не может быть отрицательным.
\end{enumerate}

    \setcounter{exercise}{34}
    \begin{exercise}[Стоимость билета в кинотеатр]
    Создайте программу определения стоимости билета в кинотеатр. Учитывайте:
    \begin{itemize}
        \item Время сеанса (до 12:00, 12:00-16:00, после 16:00)
        \item День недели (будни/выходные)
        \item Возраст посетителя (дети до 7 лет, школьники 7-17, взрослые 18+)
        \item Наличие студенческого билета
    \end{itemize}
    Базовая цена: 500 рублей. \\
    \textbf{Формула расчета:}
    $$\small \text{Итоговая цена = Базовая цена × Временной коэффициент × Коэффициент дня × Возрастной коэффициент}$$

    Где: \\
    Временной коэффициент: \\
    - До 12:00: 0.8 \\
    - 12:00-16:00: 0.9 \\
    - После 16:00: 1.0 \\
    
    Коэффициент дня: \\
    - Выходной: 1.2 \\
    - Будний: 1.0 \\
    
    Возрастной коэффициент: \\
    - До 7 лет: 0.5 \\
    - 7-17 лет: 0.7 \\
    - 18+ со студенческим: 0.8 \\
    - 18+ без студенческого: 1.0 \\
    
    \begin{inputformat}[title=Входные данные]
        time (int): время сеанса (часы, 0-23) \\
        is\_weekend (bool): выходной день \\
        age (int): возраст посетителя \\
        has\_student\_card (bool): наличие студенческого билета \\
    \end{inputformat}
        
    \begin{outputformat}[title=Пример 1]
        Входные данные: \\
        time = 10 (до 12:00) \\
        is\_weekend = $False$ \\
        age = 15 \\
        has\_student\_card = $False$ \\

        Расчет: \\
        1. Базовая цена: 500 руб. \\
        2. Временной коэффициент: ×0.8 (утро) \\
        3. День недели: ×1.0 (будни) \\
        4. Возрастной коэффициент: ×0.7 (школьник) \\
        \\ 
        500 × 0.8 × 1.0 × 0.7 = 280 руб.
    \end{outputformat}
    
    \begin{outputformat}[title=Пример 2]
        time = 19 (после 16:00) \\
        is\_weekend = $True$ \\
        age = 20 \\
        has\_student\_card = $True$ \\
    
        Расчет: \\
        1. Базовая цена: 500 руб. \\
        2. Временной коэффициент: ×1.0 (вечер) \\
        3. День недели: ×1.2 (выходной) \\
        4. Возрастной коэффициент: ×0.8 (студент) \\
        \\ 
        500 × 1.0 × 1.2 × 0.8 = 480 руб.
    \end{outputformat}
    
    
    \end{exercise}

    \begin{exercise}[Расчет стоимости доставки]
        Создайте программу расчёта стоимости доставки посылки:
    Параметры:
        \begin{itemize}
            \item Вес (до 1кг, 1-5кг, 5-10кг, более 10кг)
            \item Расстояние (до 5км, 5-20км, более 20км)
            \item Срочность (обычная/экспресс)
            \item Хрупкость груза (да/нет)
            \item Тип доставки (дверь-дверь/до пункта выдачи)
        \end{itemize}
        \textbf{Формула расчета:} \\
        
        {\small Итоговая цена = (Базовая цена + Надбавка за вес + Надбавка за расстояние) × \linebreak × 
                Коэффициент срочности × Коэффициент хрупкости + 
                Надбавка за доставку до двери} \\

Где: \\
Базовая цена = 300 руб. \\

Надбавка за вес: \\
- До 1 кг: +100 руб. \\
- 1-5 кг: +200 руб. \\
- 5-10 кг: +300 руб. \\
- Более 10 кг: +500 руб. \\

Надбавка за расстояние: \\
- До 5 км: +100 руб. \\
- 5-20 км: +200 руб. \\
- Более 20 км: +300 руб. \\

Коэффициент срочности: \\
- Экспресс: ×1.5 \\
- Обычная: ×1.0

Коэффициент хрупкости: \\
- Хрупкий: ×1.3 \\
- Обычный: ×1.0 

Надбавка за доставку до двери: +200 руб. 

        \begin{inputformat}[title=Входные данные]
            weight (float): вес в кг \\
            distance (float): расстояние в км \\
            is\_express (bool): экспресс доставка \\
            is\_fragile (bool): хрупкий груз \\
            door\_delivery (bool): доставка до двери \\
        \end{inputformat}
            
        \begin{outputformat}[title=Пример 1]
            Входные данные: \\
            weight = 0.5 (кг) \\
            distance = 3 (км) \\
            is\_express = $False$ \\
            is\_fragile = $False$ \\
            door\_delivery = $True$ \\

            Расчет: \\
            1. Базовая цена: 300 руб. \\
            2. Надбавка за вес: +100 руб. (до 1 кг) \\
            3. Надбавка за расстояние: +100 руб. (до 5 км) \\
            4. Доставка до двери: +200 руб. \\

            300 + 100 + 100 + 200 = 700 руб. 
        \end{outputformat}

    \end{exercise}



\end{document}