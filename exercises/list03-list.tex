\documentclass[a4,12pt]{article}
\usepackage[utf8]{inputenc}
\usepackage[T2A]{fontenc}
\usepackage[english, russian]{babel}


\usepackage{csquotes}
\usepackage{epigraph}
\usepackage{tcolorbox}
% Определяем окружение для Формата ввода
\newtcolorbox{inputformat}[1][]{colback=blue!5!white, colframe=blue!65!black,
    fonttitle=\bfseries, title=Формат ввода, #1}

% Определяем окружение для Формата вывода
\newtcolorbox{outputformat}[1][]{colback=green!5!white, colframe=green!65!black,
    fonttitle=\bfseries, title=Формат вывода, #1}

% Определяем окружение для комментария
\newtcolorbox{exercisecomment}[1][]{colback=yellow!5!white, colframe=yellow!80!black, fonttitle=\bfseries, title=Комментарий, #1}

% Определяем окружение для замечания
\newtcolorbox{exercisenote}[1][]{colback=red!5!white, colframe=red!80!black, fonttitle=\bfseries, title=Замечание, #1}


\usepackage{geometry}
%\geometry{papersize={20.3 cm,25.4 cm}}
\geometry{left=2cm}
\geometry{right=2cm}
\geometry{top=2cm}
\geometry{bottom=2cm}


\usepackage{amsmath, amsfonts, amsthm, amssymb, amsopn, amscd}
\usepackage{enumerate}
\usepackage{enumitem}
\usepackage[mathscr]{eucal}

\usepackage{hyperref}
\hypersetup{unicode=true,final=true,colorlinks=true}

\theoremstyle{remark}
%\newtheorem{exercises}{Упражнения}
\newtheorem{exercise}{\textbf{Упражнение}}[section]
\renewcommand{\theexercise}{\textbf{\arabic{exercise}}}
%\renewcommand{\theexercise}{\textbf{\#\arabic{exercise}}}


\title{Листок 03. Методы работы со списками и списковыми включениями в Python}

\author{А.Н. Лата}

\begin{document} 

%\maketitle

\section*{\centering Листок 03. Методы работы со списками и списковыми включениями в Python}

\begin{exercisenote}[title=Замечания]
\begin{itemize}
    \item для решение приведенных ниже упражнений не требуется создавать (определять) пользовательские функции.
    \item при решении упражнений полученные результаты выведите на экран с помощью функции {\color{blue}{print()}}.
    \item позаботесь о том, чтобы выводимый на экран результат был снабжен информацией о нем, там где это необходимо.
    \item не забывайте писать комментарии к вашему коду
\end{itemize}
\end{exercisenote}

\textbf{Упражнения}
\begin{enumerate}
    \item Создайте список из первых 5 букв русского алфавита.
    \item Создайте список из букв слова 'python'.
    \item Создайте список из первых 30 натуральных чисел. \textbf{Подсказка:} используйте функцию \textbf{range()}.
    \item Создайте список из чётных чисел от 0 до 30.
    \item Создайте список из нечётных чисел от 0 до 30.
    \item Создайте список из чисел от -5 до 5.
    \item Создайте список из чисел от 1 до 10 и добавьте в конец число 11.
    \item Создайте список из букв 'a', 'b', 'c' и вставьте букву 'd' на позицию 1.
    \item Найдите индекс элемента 'cherry' в списке ['apple', 'banana', 'cherry'].
    \item Создайте список чисел от 1 до 5 и найдите его длину.
    \item Объедините два списка [1, 2, 3] и [4, 5, 6].
    \item Создайте список из элементов [1, 2, 3], повторенных 3 раза.
    \item Объедините два отсортированных списка [1, 3, 5] и [2, 4, 6] в один отсортированный список.
    \item Удалите последний элемент из списка [1, 2, 3, 4, 5].
    \item Вставьте число 0 в начало списка [1, 2, 3].
    \item Найдите сумму элементов списка [1, 2, 3, 4, 5].
    \item Создайте список из 5 элементов и проверьте, содержится ли число 3 в списке. \textbf{Подсказка:} используйте оператор \textbf{in}.
    \item Создайте список чисел от 1 до 10 и найдите сумму всех элементов.
    \item Найдите максимальное и минимальное значения в списке [10, 5, 8, 3, 7].
    \item Найдите разницу между максимальным и минимальным элементами списка [10, 5, 8, 3, 7].
    \item Посчитайте количество элементов в списке [1, 2, 3, 4, 5].
    \item Создайте список из чисел и найдите количество элементов между первым и последним элементом.
    \item Отсортируйте список [3, 1, 4, 1, 5, 9, 2, 6, 5] по возрастанию.
    \item Переверните порядок элементов в списке [1, 2, 3, 4, 5].
    \item Поменяйте порядок элементов в списке [1, 2, 3, 4, 5] на обратный.
    \item Создайте список из чисел и отсортируйте его без изменения оригинала.
    \item Замените второй элемент списка [1, 2, 3, 4, 5] на число 10.
    \item Создайте копию списка [1, 2, 3] и измените первый элемент копии на 10.
    \item Найдите индекс первого вхождения числа 3 в списке [1, 2, 3, 4, 3, 5].
    \item Создайте список из чисел и посчитайте количество элементов, равных 5.
    \item Посчитайте количество вхождений числа 2 в список [1, 2, 2, 3, 2, 4, 2, 5].
    \item Найдите индекс первого вхождения числа 7 в список [5, 3, 7, 1, 7, 9].
    \item Создайте список из чисел и найдите индекс последнего вхождения числа 5.
    \item Создайте список из чисел и найдите количество элементов, равных минимальному значению.
    \item Объедините все строки в списке ['Hello', 'World', '!'] в одну строку.
    \item Создайте список из строк и объедините их в одну строку, разделяя пробелами.
    \item Поменяйте местами первый и последний элементы в списке [1, 2, 3, 4, 5].
    \item Поменяйте местами минимальный и максимальный элементы в списке [4, 2, 9, 7, 5, 1, 8].
    \item Создайте список из 5 элементов и удалите все элементы, используя метод clear().
    \item Создайте список из чисел и удалите элемент по индексу 3.
    \item Создайте список чисел от 1 до 10 и удалите последний элемент.
    \item Удалите из списка [1, 2, 3, 4, 5] элементы со второго по четвертый включительно.
    \item Удалите из списка [1, 2, 3, 4, 5] каждый второй элемент.
    \item Удалите из списка [1, 2, 3, 4, 5, 6, 7, 8, 9, 10] каждый третий элемент.
    \item Удалите из списка ['a', 'b', 'c', 'd', 'e'] элементы с четными индексами.

\end{enumerate}

\end{document}