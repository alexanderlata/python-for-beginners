\documentclass[a4,12pt]{article}
\usepackage[utf8]{inputenc}
\usepackage[T2A]{fontenc}
\usepackage[english, russian]{babel}


\usepackage{csquotes}
\usepackage{epigraph}
\usepackage{tcolorbox}
% Определяем окружение для Формата ввода
\newtcolorbox{inputformat}[1][]{colback=blue!5!white, colframe=blue!65!black,
    fonttitle=\bfseries, title=Формат ввода, #1}

% Определяем окружение для Формата вывода
\newtcolorbox{outputformat}[1][]{colback=green!5!white, colframe=green!65!black,
    fonttitle=\bfseries, title=Формат вывода, #1}

% Определяем окружение для комментария
\newtcolorbox{exercisecomment}[1][]{colback=yellow!5!white, colframe=yellow!80!black, fonttitle=\bfseries, title=Комментарий, #1}

% Определяем окружение для замечания
\newtcolorbox{exercisenote}[1][]{colback=red!5!white, colframe=red!80!black, fonttitle=\bfseries, title=Замечание, #1}


\usepackage{geometry}
%\geometry{papersize={20.3 cm,25.4 cm}}
\geometry{left=2cm}
\geometry{right=2cm}
\geometry{top=2cm}
\geometry{bottom=2cm}


\usepackage{amsmath, amsfonts, amsthm, amssymb, amsopn, amscd}
\usepackage{enumerate}
\usepackage{enumitem}
\usepackage[mathscr]{eucal}

\usepackage{hyperref}
\hypersetup{unicode=true,final=true,colorlinks=true}

\theoremstyle{remark}
%\newtheorem{exercises}{Упражнения}
\newtheorem{exercise}{\textbf{Упражнение}}[section]
\renewcommand{\theexercise}{\textbf{\arabic{exercise}}}
%\renewcommand{\theexercise}{\textbf{\#\arabic{exercise}}}


\title{Листок 04. Методы работы с кортежами в Python}

\author{А.Н. Лата}

\begin{document} 

%\maketitle

\section*{\centering Листок 04. Методы работы с кортежами в Python}

\begin{exercisenote}[title=Замечания]
\begin{itemize}
    \item для решение приведенных ниже упражнений не требуется создавать (определять) пользовательские функции.
    \item при решении упражнений полученные результаты выведите на экран с помощью функции {\color{blue}{print()}}.
    \item позаботесь о том, чтобы выводимый на экран результат был снабжен информацией о нем, там где это необходимо.
    \item не забывайте писать комментарии к вашему коду
\end{itemize}
\end{exercisenote}

\textbf{Упражнения}
\begin{enumerate}
    \item Создайте пустой кортеж и выведите его.
    \item Создайте кортеж с элементами: 1, 2, 3.
    \item Создайте кортеж с одним элементом 5. \textbf{Обратите внимание на синтаксис.}
    \item Создайте кортеж с одним элементом 'Hello' и выведите его тип.
    \item Создайте кортеж из строк: 'яблоко', 'банан', 'вишня'.
    \item Создайте кортеж с элементами: 1, 2, 3, 'Python', True и выведите его.
    \item Создайте кортеж с элементами из диапазона от 0 до 4. Используйте функцию \textbf{range()}.
    \item Создайте кортеж из десяти элементов и замените его на новый кортеж с другими элементами.
    \item Создайте кортеж, состоящий из символов строки `'python'`. Используйте функцию \textbf{tuple()}.
    \item Преобразуйте список [4, 5, 6] в кортеж и выведите его.
    \item Определите, какой элемент первым встречается в кортеже (5, 3, 5, 2).
    \item Создайте кортеж ('a', 'b', 'c', 'd') и выведите второй элемент.
    \item Создайте кортеж ('x', 'y', 'z') и выведите его второй элемент.
    \item Создайте кортеж (1, 2, 3, 4, 5) и выведите элемент с индексом -2.
    \item Создайте кортеж из двух чисел и вычислите их разность.
    \item Создайте кортеж (10, 20, 30, 40, 50) и выведите срез с индексами от 1 до 3 включительно.
    \item Создайте кортеж (1, 3, 5, 7, 9) и выведите срез с индексами от 2 до конца.
    \item Создайте кортеж ('apple', 'banana', 'cherry') и выведите последний элемент.
    \item Создайте кортеж (1, 2, 3, 4, 5) и выведите первые три элемента.
    \item Получите подкортеж с элементами с индексами 1 и 2 из кортежа (10, 20, 30, 40).
    \item Создайте кортеж (1, 2, 3, 4, 5) и выведите его в обратном порядке. \textbf{Используйте срезы.}
    \item Создайте кортеж чисел (1, 3, 5, 7, 9) и выведите элементы с четными индексами.
    \item Создайте кортеж из элементов ('a', 'b', 'c', 'd') и получите подкортеж ('b', 'c').
    \item В кортеже (1, 2, 2, 3, 4, 2) найдите количество вхождений числа 2.
    \item В кортеже (10, 20, 30, 40) найдите индекс элемента 30.
    \item Создайте вложенный кортеж (1, (2, 3), (4, 5, 6)) и выведите элемент 3.
    \item Объедините кортежи (1, 2, 3) и (4, 5, 6) и выведите результат.
    \item Создайте кортеж (1, 2, 3) и добавьте к нему элемент 4 с помощью конкатенации.
    \item Повторите кортеж (1, 2, 3) дважды и выведите результат.
    \item Проверьте, содержится ли элемент 'b' в кортеже ('a', 'b', 'c').
    \item Найдите длину кортежа (3, 1, 4, 1, 5) с помощью функции \textbf{len()}.
    \item Найдите минимальный и максимальный элементы в кортеже (3, 1, 4, 1, 5).
    \item Создайте кортеж чисел (5, 10, 15, 20, 25) и найдите разницу между максимальным и минимальным значениями. Выведите результат.
    \item Найдите сумму элементов кортежа (10, 20, 30).
    \item Создайте кортеж чисел (10, 20, 30, 40, 50) и найдите среднее значение элементов без использования встроенных функций. Выведите результат.
    \item Отсортируйте кортеж (5, 3, 1, 4, 2) и выведите отсортированный список.
    \item Создайте кортеж чисел `(7, 2, 5, 1, 3)` и отсортируйте его элементы по убыванию. Используйте функцию \textbf{sorted(tuple, reverse=True)}.
    \item Создайте кортеж с элементами (1, 'two', 3.0, False) и преобразуйте его в список.
    \item Преобразуйте кортеж (1, 2, 3) в список, добавьте элемент 4 и снова преобразуйте в кортеж.
    \item Создайте кортеж с числами (2, 4, 6, 8) и преобразуйте его в список. Добавьте число 10 в список и преобразуйте обратно в кортеж. Выведите итоговый кортеж.
    \item Создайте кортеж ('Python', 'Java', 'C++') и замените 'Java' на 'JavaScript' путем преобразования кортежа в список. Выведите обновленный кортеж.
    \item Создайте список [1, 2, 3], преобразуйте его в кортеж и выведите.
    \item Создайте кортеж ('Hello', 'World') и объедините его элементы в одну строку с пробелом.
    \item Создайте кортеж ('apple', 'banana', 'cherry') и преобразуйте его в строку, разделенную дефисами. Выведите результат.
    \item Создайте кортеж ('apple', 'banana', 'cherry') и выведите его элементы, разделенные запятой и пробелом.
    \item Создайте кортеж (1, 2, 3) и преобразуйте его в строку '(1, 2, 3)'.
    \item Создайте кортеж из строк 'one', 'two', 'three' и преобразуйте его в строку с помощью метода \textbf{join()}.
    \item Преобразуйте строку 'Hello, World!' в кортеж символов и выведите его.
    \item Создайте кортеж (1, 2, 3, 4, 5) и проверьте, больше ли первый элемент второго.
    \item Создайте кортеж ('a', 'b', 'a', 'c') и найдите индекс второго вхождения 'a'.
    \item Создайте кортеж ('x', 'y', 'z') и используйте распаковку кортежа для присвоения значений переменным a, b, c. Выведите переменные.
    \item Создайте кортеж чисел (1, 2, 3, 4, 5) и выведите каждый элемент на отдельной строке с помощью распаковки.
    \item Создайте кортеж ('red', 'green', 'blue', 'green') и удалите первое вхождение 'green'. Поскольку кортежи неизменяемы, создайте новый кортеж без этого элемента. Выведите результат.
    \item Создайте кортеж, содержащий элементы из списка [1, 2, 3] и строки 'abc'. Объедините их в один кортеж.
    \item Создайте кортеж из трех кортежей и найдите длину основного кортежа.




















\end{enumerate}

\end{document}