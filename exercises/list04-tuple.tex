\documentclass[a4,12pt]{article}
\usepackage[utf8]{inputenc}
\usepackage[T2A]{fontenc}
\usepackage[english, russian]{babel}


\usepackage{csquotes}
\usepackage{epigraph}
\usepackage{tcolorbox}
% Определяем окружение для Формата ввода
\newtcolorbox{inputformat}[1][]{colback=blue!5!white, colframe=blue!65!black,
    fonttitle=\bfseries, title=Формат ввода, #1}

% Определяем окружение для Формата вывода
\newtcolorbox{outputformat}[1][]{colback=green!5!white, colframe=green!65!black,
    fonttitle=\bfseries, title=Формат вывода, #1}

% Определяем окружение для комментария
\newtcolorbox{exercisecomment}[1][]{colback=yellow!5!white, colframe=yellow!80!black, fonttitle=\bfseries, title=Комментарий, #1}

% Определяем окружение для замечания
\newtcolorbox{exercisenote}[1][]{colback=red!5!white, colframe=red!80!black, fonttitle=\bfseries, title=Замечание, #1}


\usepackage{geometry}
%\geometry{papersize={20.3 cm,25.4 cm}}
\geometry{left=2cm}
\geometry{right=2cm}
\geometry{top=2cm}
\geometry{bottom=2cm}


\usepackage{amsmath, amsfonts, amsthm, amssymb, amsopn, amscd}
\usepackage{enumerate}
\usepackage{enumitem}
\usepackage[mathscr]{eucal}

\usepackage{hyperref}
\hypersetup{unicode=true,final=true,colorlinks=true}

\theoremstyle{remark}
%\newtheorem{exercises}{Упражнения}
\newtheorem{exercise}{\textbf{Упражнение}}[section]
\renewcommand{\theexercise}{\textbf{\arabic{exercise}}}
%\renewcommand{\theexercise}{\textbf{\#\arabic{exercise}}}


\title{Листок 04. Методы работы с кортежами в Python}

\author{А.Н. Лата}

\begin{document} 

%\maketitle

\section*{\centering Листок 04. Методы работы с кортежами в Python}

\begin{exercisenote}[title=Замечания]
\begin{itemize}
    \item для решение приведенных ниже упражнений не требуется создавать (определять) пользовательские функции.
    \item при решении упражнений полученные результаты выведите на экран с помощью функции {\color{blue}{print()}}.
    \item позаботесь о том, чтобы выводимый на экран результат был снабжен информацией о нем, там где это необходимо.
    \item не забывайте писать комментарии к вашему коду
\end{itemize}
\end{exercisenote}

\textbf{Упражнения}
\begin{enumerate}
    \item 1
    \item 2
\end{enumerate}

\end{document}